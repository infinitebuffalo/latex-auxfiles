% !TEX TS-program = pdflatex
% !TEX encoding = UTF-8 Unicode

\documentclass{article} 

\newcommand\ext[1]{\texttt{#1}}
\newcommand\pkg[1]{\textsf{#1}}
\newcommand\EXT[3]{\noindent #1\quad\ext{#2}\quad---\quad#3\par}
\def\<#1>{$\langle$\textit{\rmfamily n}$\rangle$}

\title{Comprehensive list of \LaTeX\ file types}
\author{Will Robertson \textit\& \texttt{comp.text.tex}}
\date{}

\begin{document}
\maketitle

\section{Introduction}

\LaTeX\ uses auxiliary files in many forms to store information between typesetting runs. This document contains a comprehensive listing covering all known packages that create such files. 

Use something like this:\\\verb|find . *.sty  -exec grep -H '\\openout.*\\jobname' {} \;|

\section{Other file extensions used by \LaTeX}

\EXT{LaTeX}{.tex}{Generic \TeX/\LaTeX\ file}
\EXT{LaTeX}{.ltx}{Generic \LaTeX\ file}
\EXT{LaTeX}{.cls}{\LaTeX\ class file}
\EXT{LaTeX}{.sty}{\LaTeX\ package file}
\EXT{LaTeX}{.dtx}{`docstrip' file for packaging code}
\EXT{LaTeX}{.ins}{Installer file for unpacking (usually) \ext{.dtx} files}
\EXT{LaTeX}{.clo}{Font size definitions, etc., for \LaTeX\ classes}

\EXT{LaTeX}{.fd}{Font definitions}
\EXT{LaTeX}{.fdd}{Documented source code for .fd files}

\EXT{}{.def}{Definition file (usually/always in reference to a font encoding)}
\EXT{}{.cfg}{Configuration file}

\EXT{}{.ldf}{babel language definition}

\EXT{inputenc}{.dfu}{Encoding definition file for UTF-8}
\EXT{latexbug}{.msg}{LaTeX bug report}

.tfm, .pk, .gf, .pl, .mf,
.vf, .vpl, .pfb, .afm, .map, .enc, .fd, .mtx, .etx

\EXT{mfpic,emp}{.mp}{MetaPost graphic}
\EXT{mfpic}{.mf}{MetaFont `graphic'}

\section{memoir}

Memoir outputs the normal jn.log, jn.aux, jn.toc, jn.lof, jn.lot files (where jn is the \@jobname) plus *.idx for index items (* is normally \@jobname but users can declare extra .idx files) plus *.gls for glossary items (* is normally \@jobname but users can declare extra .glo files) plus jn.ent for end notes.

Memoir assumes jn.aux, jn.toc, jn.lof, jn.lot as the normal input files. It takes *.ind, *.gls and jn.ent as index, glossary, and endnote input files.

Memoir's glossary system is such that MakeIndex can be used to sort the entries. The manual suggests a *.gst file for controlling MakeIndex for the glossary items (analagous to *.ist for index items).

Memoir also supplies a \verb|\newlistof| command where the user specifies the particular file extension.


\section{The list}

\EXT{kernel}{.aux}{Store generic information, mostly labels and things}
\EXT{kernel}{.log}{Complete diagnostics reported by the \LaTeX\ run.}
\EXT{kernel}{.out}{\pkg{hyperref} PDF bookmarks}
\EXT{kernel}{.toc}{Table of contents.}
\EXT{kernel}{.lot}{List of tables.}
\EXT{kernel}{.lof}{List of figures.}

\EXT{???}{.loa}{List of algorithms}

\EXT{asymptote}{.pre}{Used to communicate preamble declarations from an \pkg{Asymptote} figure to the document}

\EXT{attachfile2}{.atfi}{Attributes for attached files}

\EXT{Bib\TeX}{.blg}{}
\EXT{backref}{.brf}{Bibliographical back referencing}
\EXT{Bib\TeX}{.bbl}{Generated bibliography}

\EXT{beamer}{.nav}{}
\EXT{beamer}{.snm}{}
\EXT{beamer}{.vrb}{If the frame option fragile is set, \verb and the verbatim environment will write the fragile 
content in \backslashjobname.vrb. Source: Herbert Voß, Präsentationen mit LaTeX 1st Edition page 73}

\EXT{bookmark}{.out.ps}{Outlines as pdfmarks}

\EXT{bibref}{.bdx}{ ?}
\EXT{bibref}{.bnd}{ ?}

\EXT{gmdoc}{.auxx}{Distinct auxiliary file used by the \pkg{gmdoc} package to avoid conflicts with the \ext{aux} file}

\EXT{hypdoc}{.hd}{Used destinations}

\EXT{hyperref}{.out}{Outlines}

\EXT{}{.idv}{}
\EXT{}{.lg}{}

\EXT{}{.url}{}

\EXT{biblatex}{-blx.bib}{`Auxiliary' bibliography generated by \pkg{biblatex}.}

\EXT{endfloat}{.ttt}{Floating tables to be moved to the end of the document}
\EXT{endfloat}{.fff}{Floating figures to be moved to the end of the document}
\EXT{endnotes}{.ent}{Document notes to be moved to the end of the document}

\EXT{glossary}{.glo}{Generated glossary?}
\EXT{glossary}{.gls}{makeindex glossary?}
\EXT{glossaries}{.gls,.glo,.glg,.acr,.acn,.alg}{}

\EXT{ifplatform}{.w18}{Temporary file (should never be seen) for detecting the behaviour of the shell}

\EXT{makeindex}{.idx}{Generated index?}
\EXT{makeindex}{.ind}{makeindex index?}
\EXT{makeindex}{.ilg}{makeindex log file}

\EXT{minitoc}{.mtc}{Temporary work file.}
\EXT{minitoc}{.mtc\<n>}{Mini table of contents for a chapter.}
\EXT{minitoc}{.mlf\<n>}{Mini list of figures for a chapter.}
\EXT{minitoc}{.mlt\<n>}{Mini list of tables for a chapter.}
\EXT{minitoc}{.ptc\<n>}{Mini table of contents for a part.}
\EXT{minitoc}{.plf\<n>}{Mini list of figures for a part.}
\EXT{minitoc}{.plt\<n>}{Mini list of tables for a part.}
\EXT{minitoc}{.stc\<n>}{Mini table of contents for a section.}
\EXT{minitoc}{.slf\<n>}{Mini list of figures for a section.}
\EXT{minitoc}{.slt\<n>}{Mini list of tables for a section.}
\EXT{minitoc}{.maf}{List of discardable auxiliary files.}
\EXT{minitoc}{.mld}{Language definition file.}
\EXT{minitoc}{.mlo}{Language object file.}

These files contain the language dependant titles for the
mini-tables; if the titles are not representable in ASCII,
the .mld file loads a .mlo file (this is the case for some
languages loque chinese, farsi, hangul and hanja (korean),
japanese, some versions of malayalam and russian, and for
thai. The minitoc package provides a lot of .mld and .mlo
files, but a user can provide new ones.

\EXT{mwrite}{.com}{? The mwrite package seems broken.}
\EXT{ntheorem}{.thm}{List of theorems (and similar constructions)}

\EXT{pdfcrop}{-crop.pdf}{Cropped pdf file}

\EXT{pstool}{pstool-statusfile.txt}{Temporary file (should never be seen) for evaluating the completion status of external shell executions}

\EXT{tex4ht}{.xref}{}
\EXT{tex4ht}{.4ct}{}
\EXT{tex4ht}{.4tc}{}

\EXT{thumbpdf}{.tpt}{Thumbnail data file for pdfTeX}
\EXT{thumbpdf}{.tpm}{Thumbnail data file for pdfmark}
\EXT{thumbpdf}{.tno}{Optional thumbnails}
I forgot about \jobname.ttb -> flowfram's thumbtab file.

\EXT{WARMreader}{.wrm}{}

\EXT{pdfsync}{.pdfsync}{Deprecated; see \ext{synctex}}
\EXT{synctex}{.synctex.gz}{Synchronisation data between the text source and typeset output}

\EXT{pst-pdf}{-pics.pdf}{PDF container}
\EXT{pst-pdf}{-pics-crop.pdf}{PDF container with cropped pages}

\EXT{}{.tmp}{Temporary file}
\EXT{}{.dat}{Data file}

\EXT{slidesec}{.los}{List of slides}

\EXT{QCM}{.frm}{Generated QCM form.}
\EXT{QCM}{.msk}{Generated QCM mask.}

\EXT{apa.cls, endfloat}{.fff}{???}
\EXT{apa.cls, endfloat}{.ttt}{???}
\EXT{talk.cls}{.ttc}{???}
\EXT{edmargin}{.emd}{???}
\EXT{edmargin}{.enx}{???}
\EXT{edmargin}{.ent}{???}
\EXT{auto-pst-pdf}{-autopp.*}{Files generated via an auxiliary compilation (can be safely deleted)}
\EXT{songbook}{.aIdx}{???}
\EXT{songbook}{.kIdx}{???}
\EXT{songbook}{.tIdx}{???}
\EXT{songbook}{.tocS}{???}
\EXT{fixme}{.lox}{List of \textsc{fixme}s}
\EXT{jurabib}{.url}{???}
\EXT{jurarsp}{.rsp.aux}{???}
\EXT{thumb}{.ovr}{???}
\EXT{dinbrief}{.lbl}{???}
\EXT{pagenote}{.ent}{???}
\EXT{poemscol}{.ctn}{???}
\EXT{poemscol}{.ent}{???}
\EXT{poemscol}{.emd}{???}
\EXT{poemscol}{.enx}{???}
\EXT{changebar}{.cb}{???}
\EXT{changebar}{.cb2}{???}
\EXT{splitbib}{.sbb}{???}
\EXT{dateiliste}{.filelist}{???}
\EXT{cweb}{.scn}{???}
\EXT{juraabbrev}{.abb}{???}
\EXT{CoverPage}{.BibTeX.txt}{???}
\EXT{pdfcomment}{.upa}{???}
\EXT{pdfcomment}{.upb}{???}

skak.sty, xskak.sty  export chess positions to *.fen
xskak.sty exports chess games to *.xsk

the ledmac package --- which i haven't used (yet) for any large
project --- writes out an .end file related to the divisions of the
document (i think) and a file with numeric extensions (i.e., .1, .2, .
3, etc.) that are connected to the number of document divisions found
in the .end file (again: i think).  particularly complex documents
might have a few other files i haven't encountered yet, and i expect
its brothers, ledpar and ledarab, have certain extensions of their
own.

there is also 'ednotes', which is another package designed for
critical editions, but i have never tried it.

\end{document}
